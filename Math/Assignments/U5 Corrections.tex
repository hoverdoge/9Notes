%NOTE
\documentclass{article}
\usepackage[margin=1.0in]{geometry}
\usepackage{titling}
\usepackage{graphicx}
\usepackage{float}
\usepackage[T1]{fontenc}

\begin{document}
	
	%%% title
	\setlength{\droptitle}{-5em}
	\title{Unit 5 Test Corrections}
	\date{}
	\author{}
	\maketitle
	
	%%% first section
	\section*{5}
	The answer is the smallest angle, because the lines on the triangles are finite, and 2 sides have to be larger combined than the last one to create a triangle. The smallest side will have the other 2 sides wrap around it, and consequently form the smallest angle at the opposite side.
	\section*{8}
	The answer is a pentagon because 1/72 the sum of the exterior angles is 5, and a pentagon has 5 diagonals. \\
	What I did wrong was incorrectly finding the amount of diagonals in a pentagon. Using the $(n(n-3))/2$ formula or just drawing it out will be beneficial next time I see this type of question. \\
	\section*{12}
	The answer is sometimes. This is because all regular polygons with at least 5 sides will have obtuse interior angles, meaning they must have acute exterior angles. However, in irregular 5+ sided polygons some angles can be acute, and such some exterior angles can be obtuse. In addition, in a triangle, you can have obtuse exterior angles. \\
	What I did wrong here was a stupid error because of stress. I confused exterior angles and interior angles somehow, even though it would still be sometimes! \\
	\section*{13}
	The answer is sometimes, because the geometric mean is higher with negative numbers and lower with positive numbers. \\
	My error was in only evaluating negative numbers in my test case. I should diversify my test case in future questions.
	\section*{14}
	
		\hskip-0.5cm\begin{tabular}{|c | c|} 
			\hline
			Statements & Reasons \\ [0.5ex] 
			\hline\hline
			$\angle K \cong PLM$  & <--  \\ 
			\hline
			$\angle M \cong \angle M$ & Reflexive \\
			\hline
			Triangle $LMP$ is similar to Triangle $LKM$ & $A$A \\
			\hline
			$LM/KM = PM/LM$ & CPCTC \\
			\hline
			$\angle LM ^ 2 = PM * KM$ & Means-Extremes Theorem \\[1ex] 
			\hline	
			\end{tabular}
		
		The way to do this is to work backwards, and realize that through the Means-Extremes theorem, you only need to get that ()$\angle LM/ \angle KM) = (\angle PM/\angle LM$). This is possible through proving the triangles' similarity.
\end{document}
