%NOTE
\documentclass{article}
\usepackage[margin=1.0in]{geometry}
\usepackage{titling}
\usepackage{graphicx}
\usepackage{float}
\usepackage[T1]{fontenc}

\begin{document}
	
	%%% title
	\setlength{\droptitle}{-5em}
	\title{Quadratic Forms}
	\date{}
	\author{}
	\maketitle
	
	%%% first section
	\section*{Quadratics}
	\textbf{General Form}: $ax^2 + bx + c$ \\
	Example: $x^2 + 6x + 9$ \\
	\textbf{$A$} determines whether the parabola opens up/down (-a is down)\\
	\textbf{$B$} moves the axis of symmetry from side to side\\
	\textbf{$C$} is the Y intercept\\ 
	\textbf{Vertex:} ($-b/2a$), ($f(-b/2a)$)\\
	\textbf{Axis of Symmetry:} $-b/2a$ \\
	\textbf{Y-Intercept:} $C$\\
	\textbf{X-Intercept:} Solve\\ \\
	\textbf{Vertex Form}: $a(x-h)^2+k$ \\
	Example: $-2(x-4)^{2}+2$\\
	$A$ determines whether the parabola opens up/down (-a is down) \\
	$(-H, K)$ is the vertex \\ 
	\textbf{Vertex:} ($-h, k$) \\
	\textbf{Axis of Symmetry:} $h$ \\
	\textbf{Y-Intercept:} Set $x=0$, solve for $y$\\
	\textbf{X-Intercept:} Set $y=0$, solve for $x$)\\ \\
	\textbf{Factored Form}: $a(x-r)(x-s)$ \\
	Example: $(x+3)(x+2)$ \\
	$A$ determines whether the parabola opens up/down (-a is down) \\
	$(-R, -S)$ are the X intercepts \\ 
	\textbf{Vertex:} Get the average of the $x$-intercepts, substitute )\\
	\textbf{Axis of Symmetry:} $(r+s)/2$\\
	\textbf{Y-intercept:} Set $x=0$, solve for $y$ \\
	\textbf{X-Intercept:} $(-r, -s)$  \\
\end{document}
