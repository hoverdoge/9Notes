\documentclass{article}
\usepackage[margin=1.0in]{geometry}
\usepackage{titling}
\usepackage{graphicx}
\usepackage{float}
\usepackage[T1]{fontenc}

\begin{document}
	
	%%% title
	\setlength{\droptitle}{-5em}
	\title{1.2 Deductive Structures}
	\date{}
	\author{}
	\maketitle
	
	%%% first section
	\section{General Statements}
	\indent \textbf{Counter Example}: An example that disproves that something is always true \newline
	\indent Example: "All prime numbers are odd" is disproved by "2 is a prime number" \newline \newline
	
	\noindent \textbf{Deductive Structure}: Scaffolding of information used for reasoning \newline
	\indent Example: $A = B$. $B = C$. Therefore, $A = C$
	\newline \newline
	
	\noindent \textbf{Postulates $\rightarrow$ Theorems $\rightarrow$ Definitions} \newline
	\indent Unproven $\rightarrow$ Proven $\rightarrow$ definition of statement/word
	\newline \newline
	
	\noindent \textbf{Declarative Statement:} Statement without "if" "then" "but"
	\newline
	\indent Example: An odd number + odd number is even
	\newline \newline
	
	\noindent \textbf{Conditional Statement:} If...Then
	\newline
	\indent Example: If P then Q
	
	%%%% CONT/INV
	\section{Types of Conditional Statements}
	
	\textbf{Normal}\newline \indent If P then Q \newline
	\textbf{Converse}\newline \indent If Q then P \newline
	\textbf{Inverse}\newline \indent If !P then !Q \newline
	\textbf{Contrapositive}\newline \indent If !Q then !P \newline
\end{document}